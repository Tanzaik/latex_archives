\documentclass[10pt,a4paper]{article}
\usepackage[latin1]{inputenc}
\usepackage{amsmath,amsfonts, amssymb, tikz}
\usepackage[hmargin=0.5in,vmargin=0.5in]{geometry}


\begin{document}
\hfill Tanzim Amin\\
\begin{enumerate}
\item A skydiver weighing 180 lbs (including equipment) falls vertically downward from an altitude of 5000 ft
and opens the parachute after 10 s of free fall. Assume that the force of air resistance, which is directed
opposite to the velocity, is of magnitude $0.75 \vert v \vert$ when the parachute is closed and is of
magnitude $12 \vert v \vert$ when the parachute is open, where the velocity \textit{v} is measured in ft/s.\\
Notation: We let $v_1(t)$ and $s_1(t)$ be the velocity and position of the skydiver before the parachute opens
and $v_2(t)$ and $s_2(t)$ be the velocity and position of the skydiver after the parachute opens.
\begin{enumerate}
\item Find the speed of the skydiver when the parachute opens.\\
To find the velocity we solve the differential equation $mv_1'=-mg-0.75v_1$ with initial condition $v_1(0)=0$
to get
\begin{equation}
v_1(t)=-240+240e^{-0.1333t}.
\end{equation}
The speed of the skydiver when the parachute opens is $\vert v_1(10)\vert = 176.7 \text{ ft/sec} $
\item Find the distance fallen before the parachute opens.\\
To find the distance we must find the position which is found by solving the differential equation
$s_1'(t) = v_1(t)$ with initial condition $s_1(0) = 5000$ where $v_1(t)$ is from (1). The solution is
\begin{equation}
s_1(t)=16800-240t-1800e^{-0.1333t}.
\end{equation}
The distance the skydiver has fallen before the parachute opens is $5000 - s_1(10) = 1074.5$ ft.
\item What is the limiting velocity $v_L$ after the parachute opens?\\
To find the limiting velocity after the parachute opens, we must solve the differential equation\\ $mv_2'=-
mg-12v_2$ with initial condition $v_2(0)=-176.7$ to get
\begin{equation}
v_2(t)=-15-23.2e^{-32t/15}
\end{equation}
then take the limit to get
\begin{equation}
 v_L = \displaystyle{\lim_{t \to \infty} v_2(t)=-15 \text{ ft/sec}}
\end{equation}
\item Determine how long the skydiver is in the air after the parachute opens.\\
We find the position after the parachute opens by solving the differential equation
\begin{equation}
s_2'(t)=v_2(t) \hspace{0.5cm}s_2(0)=s_1(10)=3925.4
\end{equation}
where $v_2$ is from (3) to get the solution
\begin{equation}
s_2(t)=3849.72-15t+75.814e^{-32t/15}
\end{equation}
and solve $s_2(t)=0$ to get $t=256.6$ seconds which implies that the person is in the air for 266.6 seconds.
1
\item Plot the graph of velocity versus time and position versus time from the beginning of the fall until the
skydiver reaches the ground.\\
\includegraphics[scale=0.2]{VELOCITYGRAPH.png}}
\includegraphics[scale=0.2]{PARACHUTE GRAPH.png}}
\end{enumerate}
\pagebreak
\item A tank that has a capacity of 50 gallons is initially filled with 20 gallons of liquid where 2 lbs of
salt has been dissolved. At time $t=0$ a solution that has a salt concentration of $\frac{1}{2}$ lb/gal
enters the tank at a rate of 4 gal/min. At the same time the well mixed solution starts draining from the
tank at 3 gal/min.\\
Notation: Let $A_1(t)$ be the amount of salt (in lbs) in the tank before it fills up and $A_2(t)$ be the
amount of salt in the tank after it starts to overflow. Time is measured in minutes.
\begin{enumerate}
\item Derive the equation that described the amount of salt in the tank at any time before the tank fills up.\
\
We solve the differential equation $\frac{dA_1}{dt}=2-\frac{3A_1}{20+t}$, where $A_1(0)=2$, to get
\begin{equation}
A_1(t)=\dfrac{37+t}{2}-\dfrac{835774.5}{(37+t)^3}
\end{equation}
\item How much salt is in the tank at the time when the tank becomes full?\\
The tank fills up at $t=63$ minutes so there will be $A_1(63)=99.1$ lbs of salt.\\
\item If the tank is then allowed to overflow, derive the equation that described the amount of salt in the
tank at any time after it has started to overflow.\\
Now we solve the differential equation $\frac{dA_2}{dt}=2-\frac{4A_2}{50}$, were $A_2(0)=A_1(63)=-10.44$ to
get
\begin{equation}
A_2(t)=50-10.44e^{-2t/25}
\end{equation}
\item How much salt is in the tank at time 100 minutes after the problem starts? \\
Since the tank fills up after 63 minutes, the amount of salt will be $A_2(37)=47.62$ lbs.\\
\item Create a beautiful graph done in Mathematica that shows the amount of salt in the tank for values of t
from 0 to 100. \\
\begin{center}
\includegraphics[scale=0.7]{TANKGRAPH.png}
\end{center}
\end{enumerate}
\end{enumerate}
\end{document}
